% Homework template for Inference and Information
% UPDATE: September 26, 2017 by Xiangxiang
\documentclass[a4paper]{article}
\usepackage{ctex}
\usepackage{amsmath, amssymb, amsthm}
\usepackage{moreenum}
\usepackage{mathtools}
\usepackage{url}
\usepackage{bm}
\usepackage{enumitem}
\usepackage{graphicx}
\usepackage{listings}
\usepackage{color}

\lstset{
    basicstyle          =   \sffamily,          % 基本代码风格
    keywordstyle        =   \bfseries,          % 关键字风格
    commentstyle        =   \rmfamily\itshape,  % 注释的风格,斜体
    stringstyle         =   \ttfamily,  % 字符串风格
    flexiblecolumns,                % 别问为什么,加上这个
    numbers             =   left,   % 行号的位置在左边
    showspaces          =   false,  % 是否显示空格,显示了有点乱,所以不现实了
    numberstyle         =   \zihao{-5}\ttfamily,    % 行号的样式,小五号,tt等宽字体
    showstringspaces    =   false,
    captionpos          =   t,      % 这段代码的名字所呈现的位置,t指的是top上面
    frame               =   lrtb,   % 显示边框
}

\lstdefinestyle{Python}{
    language        =   Python, % 语言选Python
    basicstyle      =   \zihao{-5}\ttfamily,
    numberstyle     =   \zihao{-5}\ttfamily,
    keywordstyle    =   \color{blue},
    keywordstyle    =   [2] \color{teal},
    stringstyle     =   \color{magenta},
    commentstyle    =   \color{red}\ttfamily,
    breaklines      =   true,   % 自动换行,建议不要写太长的行
    columns         =   fixed,  % 如果不加这一句,字间距就不固定,很丑,必须加
    basewidth       =   0.5em,
}
\usepackage{subcaption}
\usepackage{booktabs} % toprule
\usepackage[mathcal]{eucal}
\usepackage[thehwcnt = 3]{iidef}

\thecourseinstitute{清华大学电子工程系}
\thecoursename{\textbf{媒体与认知} \space 课堂2}
\theterm{2021-2022学年春季学期}
\hwname{作业}
\begin{document}
\courseheader
\name{YOUR NAME}
\vspace{3mm}
\centerline{\textbf{\Large{理论部分}}}

\section{单选题(15分)}
\subsection{\underline{?}}

\subsection{\underline{?}}

\subsection{\underline{?}}

\subsection{\underline{?}}

\subsection{\underline{?}}

\section{计算题(15 分)}
\subsection{
已知某卷积层的输入为$X$(该批量中样本数目为1,输入样本通道数为1),采用一个卷积核$W$,即卷积输出通道数为1,卷积核尺寸为$2\times 2$,卷积的步长为1,无边界延拓,偏置量为$b$:
$$X=\left[ \begin{array}{ccc}
    -0.5 & 0.2 & 0.3 \\
    0.6 & -0.4 & 0.1 \\
    0.4 & 0.5 & -0.2
\end{array}\right],
W=\left[ \begin{array}{cc}
    -0.2 & 0.1  \\
    0.4 & -0.3
\end{array}\right], b=0.05$$
}
\subsubsection{请计算卷积层的输出$Y$。}

\subsubsection{若训练过程中的目标函数为$L$,且已知$\frac{\partial L}{\partial Y}=\left[ \begin{array}{cc}
    0.1 & -0.2 \\
    0.2 & 0.3
\end{array} \right]$,请计算$\frac{\partial L}{\partial X}$。
}

注:本题的计算方式不限,但需要提供计算过程以及各步骤的结果。
\vspace{6mm}
\centerline{\textbf{\Large{编程部分}}}
\vspace{3mm}

% 请根据是否选择自选课题的情况选择“编程作业报告”或“自选课题开题报告”中的一项完成
\section{编程作业报告}
\section{自选课题工作进度汇报}

\end{document}



%%% Local Variables:
%%% mode: late\rvx
%%% TeX-master: t
%%% End:
