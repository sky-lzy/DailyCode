% Homework template for Inference and Information
% UPDATE: September 26, 2017 by Xiangxiang
\documentclass[a4paper]{article}
\usepackage{ctex}
\usepackage{amsmath, amssymb, amsthm}
\usepackage{moreenum}
\usepackage{mathtools}
\usepackage{url}
\usepackage{bm}
\usepackage{enumitem}
\usepackage{graphicx}
\usepackage{listings}
\usepackage{color}

\lstset{
    basicstyle          =   \sffamily,          % 基本代码风格
    keywordstyle        =   \bfseries,          % 关键字风格
    commentstyle        =   \rmfamily\itshape,  % 注释的风格,斜体
    stringstyle         =   \ttfamily,  % 字符串风格
    flexiblecolumns,                % 别问为什么,加上这个
    numbers             =   left,   % 行号的位置在左边
    showspaces          =   false,  % 是否显示空格,显示了有点乱,所以不现实了
    numberstyle         =   \zihao{-5}\ttfamily,    % 行号的样式,小五号,tt等宽字体
    showstringspaces    =   false,
    captionpos          =   t,      % 这段代码的名字所呈现的位置,t指的是top上面
    frame               =   lrtb,   % 显示边框
}

\lstdefinestyle{Python}{
    language        =   Python, % 语言选Python
    basicstyle      =   \zihao{-5}\ttfamily,
    numberstyle     =   \zihao{-5}\ttfamily,
    keywordstyle    =   \color{blue},
    keywordstyle    =   [2] \color{teal},
    stringstyle     =   \color{magenta},
    commentstyle    =   \color{red}\ttfamily,
    breaklines      =   true,   % 自动换行,建议不要写太长的行
    columns         =   fixed,  % 如果不加这一句,字间距就不固定,很丑,必须加
    basewidth       =   0.5em,
}
\usepackage{subcaption}
\usepackage{booktabs} % toprule
\usepackage[mathcal]{eucal}
\usepackage[thehwcnt = 2]{iidef}

\thecourseinstitute{清华大学电子工程系}
\thecoursename{\textbf{媒体与认知} \space 课堂2}
\theterm{2021-2022学年春季学期}
\hwname{作业}
\begin{document}
\courseheader
\name{YOUR NAME}
\vspace{3mm}
\centerline{\textbf{\Large{理论部分}}}

\section{单选题(15分)}
\subsection{\underline{?}}

\subsection{\underline{?}}

\subsection{\underline{?}}

\subsection{\underline{?}}

\subsection{\underline{?}}

\section{计算题(15 分)}
\subsection{设隐含层为$\mathbf{z}=\mathbf{x}\mathbf{W}^T+\mathbf{b}$,其中$\mathbf{x}\in R^{(1 \times m)}$,$\mathbf{z}\in R^{(1\times n)}$,$\mathbf{W}\in R^{(n\times m)}$,$\mathbf{b} \in R^{(1\times n)}$均为已知,其激活函数如下:
$$\mathbf{y}=\tanh(\mathbf{z})=\frac{e^\mathbf{z}-e^{-\mathbf{z}}}{e^\mathbf{z}+e^{-\mathbf{z}}}$$
若训练过程中的目标函数为L,且已知L对$\mathbf{y}$的导数 $\frac{\partial L}{\partial \mathbf{y}}=[\frac{\partial L}{\partial y_1},\frac{\partial L}{\partial y_2},...,\frac{\partial L}{\partial y_n}]$和$\mathbf{y}=[y_1,y_2,...,y_n]$的值。
}
\subsubsection{请使用$\mathbf{y}$表示出$\frac{\partial \mathbf{y}}{\partial \mathbf{z}}$
}

\subsubsection{请使用$\mathbf{y}$和$\frac{\partial L}{\partial \mathbf{y}}$表示$\frac{\partial L}{\partial \mathbf{x}}$,$\frac{\partial L}{\partial \mathbf{W}}$,$\frac{\partial L}{\partial \mathbf{b}}$。
}
提示:$\frac{\partial L}{\partial \mathbf{x}}$,$\frac{\partial L}{\partial \mathbf{W}}$,$\frac{\partial L}{\partial \mathbf{b}}$与x,W,b具有相同维度。

\vspace{6mm}
\centerline{\textbf{\Large{编程部分}}}
\vspace{3mm}
% 请根据是否选择自选课题的情况选择“编程作业报告”或“自选课题开题报告”中的一项完成
\section{编程作业报告}
\section{自选课题工作进度汇报}

\end{document}



%%% Local Variables:
%%% mode: late\rvx
%%% TeX-master: t
%%% End:
